\documentclass[conference,a4paper]{IEEEtran}

\usepackage{cite}

\begin{document}

\title{A secondary screen architecture to accurately capture viewer interactions in interactive TV environments}

\author{\IEEEauthorblockN{Ricardo E. V. de S. Rosa}
\IEEEauthorblockA{Graduate Program in Electrical Engineering \\ Federal University of Minas Gerais \\ Av. Antônio Carlos 6627, 31270-901, \\Belo Horizonte, MG, Brazil\\
Email: ricardoerikson@ufmg.br}
\and
\IEEEauthorblockN{Vicente F. de Lucena Junior}
\IEEEauthorblockA{Graduate Program in Electrical Engineering \\
Federal University of Amazonas \\
Manaus, AM, Brazil\\
Email: vicente@ufam.edu.br}
}
\maketitle

\begin{abstract}
Watching TV is frequently seen as a social activity. Groups of people gather around the television for entertainment or information. One important application for TV viewers is content personalization, which consists in generating personalized content based on viewer interactions. Since the TV set is shared by many people, it is difficult to accurately capture the individual interactions from every viewer using a traditional remote control. As a new trend, the proliferation of personal mobile devices has changed the habits on audiovisual content consumption in interactive TV environments. In many situations people use their personal devices as a secondary screen to perform activities related to television watching. In this paper, we present an architecture for secondary screen devices that facilitates the capture of individual viewer interactions to use in content recommendation. In addition, this architecture can enhance the viewer interaction without interrupting the collective experience while watching TV.
\end{abstract}

\IEEEpeerreviewmaketitle

\section{Introduction}

Watching TV is essentially a social activity, where family, friends, or people with a common interest share the same space and TV set for entertainment or information~\cite{Masthoff2004}. After many advances in technologies for interactive and digital TV it is possible to develop high level applications to enrich the TV experience. One important application is content personalization, in which content providers deliver personalized content to viewers based on their preferences.

The viewer preferences can be obtained by observing the viewing habits as interactions between the viewers and the TV system. Such interactions can be classified as implicit and explicit. Implicit interactions are part of the natural actions of watching TV e.g., change channel and volume. In explicit interactions viewers contribute by evaluating, commenting or sharing their opinion about a given content. Considering the data of thousands of viewers, the data captured can provide valuable information for content personalization services~\cite{Teixeira2010}.

The traditional remote control as a mechanism of interaction between viewer and TV can present notable problems.It is difficult to automatically identify the viewer in charge of the remote control. Thus, in practice the captured interactions are assigned to the whole group, rather than to individual viewers. As a result, in recommender systems applications for example, the tastes of the user in charge of the remote control can be imposed on the tastes of others.

Considering the collective experience of watching TV, the explicit interactions of one viewer can potencialy interfere in the experience of other viewers. Since there is only one remote control and a unique screen (TV screen), explicit interactions that require the use this screen (e.g., evaluate the current content or read viewers comments) end up in an undesired experience for users that are not actively participating of that interactions.

The proliferation of personal mobile devices has changed the people behaviors towards TV. Frequently people perform secondary tasks while watching TV such as use of email, use of social networks or surf the web. Many times these tasks are related to television watching where the viewer can evaluate, share or even recommend content to other people. Following this trend, more and more TV providers are enabling the audience to use personal devices as an extension of TV, i.e., as a secondary screen to enrich and immerse the user into the TV experience.

Identifying the viewers and capturing their interactions in TV environments are challenging activities~\cite{Hwang2007,Teixeira2010}. 

to perform secondary tasks while watching TV

These tasks are frequently related to television watching e.g., evaluating content according to personal tastes, sharing, or even recommending content to other people. This paper presents an architecture that 

TV family behaviors towards television.
potential use of mobile devices to capture individual user interactions.

opens up new possibilities to accurately capture individual interactions from TV users. by using secondary screen devices.

\section{Architecture}

\section{Conclusions}

\bibliographystyle{IEEEtran}
\bibliography{biblio.bib}



\end{document}


